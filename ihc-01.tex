\documentclass[11pt]{beamer}
\usepackage{listings} % Include the listings-package
\usepackage[T1]{fontenc}
\usepackage[utf8]{inputenc}
\usepackage[english]{babel}
\usepackage{amsmath}
\usepackage{amssymb, amsfonts, latexsym, cancel}
\usepackage{float}
\usepackage{graphicx}
\usepackage{epstopdf}
\usepackage{subfigure}
\usepackage{hyperref}
%\usepackage{authblk}
\usepackage{blindtext}
\usepackage{booktabs} % Allows the use of \toprule, 
\usepackage{filecontents}
\usepackage{courier} %% Sets font for listing as Courier.
\usepackage{listings}
%\usepackage{listings, xcolor}
\lstset{
tabsize = 2, %% set tab space width
showstringspaces = false, %% prevent space marking in strings, string is defined as the text that is generally printed directly to the console
numbers = left, %% display line numbers on the left
commentstyle = \color{green}, %% set comment color
keywordstyle = \color{blue}, %% set keyword color
stringstyle = \color{red}, %% set string color
rulecolor = \color{black}, %% set frame color to avoid being affected by text color
basicstyle = \small \ttfamily , %% set listing font and size
breaklines = true, %% enable line breaking
numberstyle = \tiny,
}
\usepackage{caption}
\DeclareCaptionFont{white}{\color{white}}
\DeclareCaptionFormat{listing}{\colorbox{gray}{\parbox{\textwidth}{#1#2#3}}}
\captionsetup[lstlisting]{format=listing,labelfont=white,textfont=white}
\definecolor{urlColor}{rgb}{0.06, 0.3, 0.57}
\definecolor{linkColor}{rgb}{0.57, 0.0, 0.04}
\definecolor{fileColor}{rgb}{0.0, 0.26, 0.26}
\hypersetup{
    colorlinks=true,
    linkcolor=linkColor,
    filecolor=fileColor,      
    urlcolor=urlColor,
}
\urlstyle{same}
\setbeamercovered{transparent}
%\usetheme{Boadilla}
\usetheme{CambridgeUS}
%\usetheme{Berkeley}
%\usetheme{Warsaw}
%\usetheme{Madrid}

\title[Introducción]{\bf\Huge Pautas de diseño de la interfaz de usuario}

\author[rescobedoq]
{
    Chirinos Sanchez Maria\inst{1}\\
	Gomez Velasco Brian Joseph \inst{2}\\
	Choqueneira Ccasa Paulina \inst{3}\\
	Olaechea Carlo Alex Williams\inst{4}
}
\institute[UNSA]
{
\inst{1}% 
System Engineering School\\
}
\date[2020-09-09]{\scriptsize{2020-09-09}}
%\logo{\includegraphics[width=3.0cm]{img/logo_unsa.jpg}}
\titlegraphic{\includegraphics[width=1.0cm]{img/logo_unsa.jpg}}

\begin{document}

\begin{frame}
\titlepage
.\end{frame}

\begin{frame}
\frametitle{Content}
\tableofcontents
\end{frame}

\section{Don Norman}
\begin{frame}
\frametitle{Don Norman}
\begin{itemize}
 \item  Donald A. Norman (25 de diciembre de 1935) es profesor emérito de ciencia cognitiva en la University of California, San Diego y profesor de Ciencias de la Computación en la Northwestern University, pero hoy en día trabaja principalmente con la ciencia cognitiva en el dominio de la ingeniería de la usabilidad. También enseña en Stanford University.

\begin{figure}[t]
\includegraphics[width=4cm, height=4cm]{norman-don.jpg}
\centering
\end{figure}
\end{itemize}
\end{frame}

\section{Aportes}
%References frame
\begin{frame}
\frametitle{Aportes}
\begin{itemize}
\item Don Norman fue profesor, investigador y autor prolífico en el campo de la psicología cognitiva mucho antes de que comenzara a escribir sobre la interacción humano-computadora. 
\item Su mayor aporte en la rama de la computación podríamos hacer referencias a las publicaciones en cuanto a la usabilidad. Entre ellos se encuentra “Emotional Design: Why we love (or hate) everyday things” y “The Design of Future Things”. Además de esto realizo algunas publicaciones en el área de psicología, entre las que se encuentran “Memory and attention” y “Learning and memory”.
\item  Según Donald Norman , hay dos principios claves para una buena interacción humano-computadora: Visibilidad y Provisión.
\end{itemize}
\end{frame}

%References frame
\begin{frame}
\frametitle{Aportes}
\begin{itemize}

\item Apesar de que las pautas de diseño propuestas por varios investigadores señalan que estas deben ser aplicadas cuidadosamente por personas expertas en el arte de la interfaz de usuario, Donald las proporciona como listas simples de edictos de diseño con poca o ninguna relación.
\item Norman incluyó los errores cognitivos en las pautas de diseño porque buscaba que alguno de estos (diseños) redujera o eliminara el impacto de los errores, que son comunes en humanos. 

\end{itemize}
\end{frame}

%References frame
\begin{frame}
\frametitle{Aportes}
\begin{itemize}
\item Los términos "gulfs of execution" y "gulfs of evaluation" fueron introducidos y popularizados por Donald. El primero se refiere al abismo existente entre lo que quiere el usuario de una herramienta y las operaciones que la herramienta proporciona, y el segundo al grado en el que la herramienta proporciona información de su estado que pueden interpretarse y emplearse directamente por el usuario.
\item Las primeras pautas de diseño humano-computadora de Norman se basaron en la investigación, la suya y la de otros, sobre la cognición humana.

\end{itemize}
\end{frame}

%References frame
\begin{frame}
\frametitle{Aportes}

\textbf{Los cuatro principios del buen diseño de Donald Norman:}
En su libro “Psicología de los objetos cotidianos” norman analiza los errores de diseño mas comunes, las causas cognitivas que están tras ellos y propone unos principios de diseño que son una referencia para cualquiera que se dedique al diseño de interacción. El libro trata bastantes conceptos y principios, que se pueden resumir en cuatro:

\begin{enumerate}
    \item Visibilidad
    \item Buena topografía
    \item Retroalimentación
    \item Buen modelo conceptual
\end{enumerate}

\end{frame}

\section{References}
%References frame
\begin{frame}
\frametitle{References}
\begin{itemize}
\item Intdrouction - xiii, Jhonson J. (2014). Designing with the Mind in mind. 2nd. edition.
\item Norman, D. A. (1998). La psicología de los objetos cotidianos (Vol. 6). Editorial Nerea.
\item Norman, D. A. (1986). Cognitive engineering. En: User centered system desing: New perspectives on human–computer interaction. Hillsdale, NJ: Erlbaum Associates.
\end{itemize}
\end{frame}

\end{document}
